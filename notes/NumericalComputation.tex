%% LyX 2.0.3 created this file.  For more info, see http://www.lyx.org/.
%% Do not edit unless you really know what you are doing.
\documentclass[twoside,english]{paper}
\usepackage{lmodern}
\renewcommand{\ttdefault}{lmodern}
\usepackage[T1]{fontenc}
\usepackage[latin9]{inputenc}
\usepackage[a4paper]{geometry}
\geometry{verbose,tmargin=3cm,bmargin=2.5cm,lmargin=2cm,rmargin=2cm}
\usepackage{color}
\usepackage{babel}
\usepackage{float}
\usepackage{bm}
\usepackage{amsthm}
\usepackage{amsmath}
\usepackage{amssymb}
\usepackage{graphicx}
\usepackage{esint}
\usepackage[unicode=true,pdfusetitle,
 bookmarks=true,bookmarksnumbered=false,bookmarksopen=false,
 breaklinks=false,pdfborder={0 0 0},backref=false,colorlinks=false]
 {hyperref}
\usepackage{breakurl}
\usepackage{mathrsfs}

\makeatletter

%%%%%%%%%%%%%%%%%%%%%%%%%%%%%% LyX specific LaTeX commands.
%% Because html converters don't know tabularnewline
\providecommand{\tabularnewline}{\\}

%%%%%%%%%%%%%%%%%%%%%%%%%%%%%% Textclass specific LaTeX commands.
\numberwithin{equation}{section}
\numberwithin{figure}{section}

%%%%%%%%%%%%%%%%%%%%%%%%%%%%%% User specified LaTeX commands.
\usepackage{babel}

\@ifundefined{showcaptionsetup}{}{%
 \PassOptionsToPackage{caption=false}{subfig}}
\usepackage{subfig}
\makeatother

\begin{document}

\section{Numerical computation of the imaginary part of a Compton form factor}

The basic integral to be computed when evaluating a Compton form
factor has the following structure:
\begin{equation}
\mathcal{H}(\xi) = \int_{-1}^{1}dx\left[C(x,\xi)-C(-x,\xi)\right]H(x)\,.
\end{equation}
Defining $H^{(+)}(x)= H(x)-H(-x)$\footnote{Using Ji's convention, the
  function $H$ can be either a quark or a gluon distribution.}, it is
easy to see that:
\begin{equation}
\mathcal{H}(\xi) = \int_{0}^{1}dx\left[C(x,\xi)-C(-x,\xi)\right]H^{(+)}(x)\,.
\end{equation}
This step is of fundamental importance because restricts the
integration to positive values of $x$ allowing, as we will see below,
for important simplifications. In addition it tuns out that the
coefficient function $C$ enjoys the following property:
\begin{equation}\label{eq:simpl}
C(x,\xi) = \frac{1}{x}\widetilde{C}\left(\frac{\xi}{x}\right)\,.
\end{equation}
This equality is actually strictly true only if one neglects the
$i\varepsilon$ prescription required to regularise the propagator
poles. However, depending on the sign of $x$, one can always reinstate
the $i\varepsilon$ prescription taking care of matching the sign of
$i\varepsilon$ with the sign of $x$. This identity allows one to
write:
\begin{equation}
\mathcal{H}(\xi) = \int_{0}^{1}\frac{dx}{x}\left[\widetilde{C}\left(\frac{\xi}{x}\right)+\widetilde{C}\left(-\frac{\xi}{x}\right)\right]H^{(+)}(x)\,,
\end{equation}
that with a simple change of variable becomes:
\begin{equation}\label{eq:strucfunc}
\mathcal{H}(\xi) = \int_{\xi}^{\infty}\frac{dy}{y}\left[\widetilde{C}\left(y\right)+\widetilde{C}\left(-y\right)\right]H^{(+)}\left(\frac{\xi}{y}\right)\,.
\end{equation}
Now, if:
\begin{equation}
C(x,\xi) = \frac{1}{x+\xi-i\varepsilon}\left(-3-2\ln\left(\frac{x+\xi}{2\xi}-i\varepsilon\right)\right)\,,
\end{equation}
using Eq.~(\ref{eq:simpl}) it is easy to see that:
\begin{equation}
\widetilde{C}(y) = \frac{1}{1+y-i\varepsilon}\left(-3-2\ln\left(\frac{1+y}{2y}-i\varepsilon \right)\right)\,,
\end{equation}
and:
\begin{equation}
\widetilde{C}(-y) = \frac{1}{1-y+i\varepsilon}\left(-3-2\ln\left(-\frac{1-y}{2y}+i\varepsilon \right)\right)\,.
\end{equation}
Notice that in $C(-y)$ we have swapped the sign of both $y$ and
$i\varepsilon$ as appropriate due to the definition of $\widetilde{C}$
in Eq.~(\ref{eq:simpl}). Since the integral in
Eq.~(\ref{eq:strucfunc}) runs over positive values of $y$,
$i\varepsilon$ inside $C(y)$ can be dropped because $C(y)$ has no pole
over the integration range. As a direct consequence $C(y)$ becomes a
real function. Since we are interested in computing the imaginary part
of $\mathcal{H}$, owing to the fact that $H$ is also a real function,
one finds that:
\begin{equation}
\mbox{Im}\left[\mathcal{H}(\xi)\right] = \int_{\xi}^{\infty}\frac{dy}{y}\mbox{Im}\left[\widetilde{C}\left(-y\right)\right]H^{(+)}\left(\frac{\xi}{y}\right)\,.
\end{equation}
Now we use the well-know relation:
\begin{equation}
\frac{1}{1-y+i\varepsilon}=\mbox{P.V.}\frac{1}{1-y}-i\pi\delta(1-y)\,,
\end{equation}
and the slightly more subtle equality:
\begin{equation}
\ln\left(-\frac{1-y}{2y}+i\varepsilon \right) = \ln\left(\left|\frac{1-y}{2y}\right|\right)-i\pi\theta(1-y)\,.
\end{equation}
The presence of the $\theta$-function in front of $i\pi$ is
consequence of the fact that that term only arises if the argument of
the logarithm is negative that only happens if $y<1$. This allows one
to write:
\begin{equation}
\mbox{Im}\left[\widetilde{C}(-y)\right] =2\pi \left[\frac{\theta(1-y)}{1-y}+\delta(1-y)\left(\frac{3}{2}+\ln\left(1-y\right)-\ln(2)\right) \right]\,.
\end{equation}
Notice that we already set $\delta(1-y)\ln y = 0$ and that, due to the
presence of the $\theta$-function, we dropped the principal
value. Despite the $\theta$-function exposes the singularity of
$1/(1-y)$, this is exactly what is needed to cancel the singularity
generated by $\delta(1-y)\ln(1-y)$. To see this write:
\begin{equation}
\delta(1-y) \ln\left(1-y\right) = -\delta(1-y)\int_0^1\frac{dz}{1-z}\,,
\end{equation}
and using the definition of $+$-prescription:
\begin{equation}
\left(\frac{1}{1-y}\right)_+ = \frac{1}{1-y}-\delta(1-y) \int_0^1\frac{dz}{1-z}\,,
\end{equation}
we finally have:
\begin{equation}
\mbox{Im}\left[\widetilde{C}(-y)\right] =2\pi \left[\theta(1-y) \left(\frac{1}{1-y}\right)_++\delta(1-y)\left(\frac{3}{2}-\ln(2)\right) \right]\,.
\end{equation}
that plugged into Eq.~(\ref{eq:strucfunc}) gives:
\begin{equation}
\begin{array}{rcl}
\displaystyle\frac{1}{2\pi}\mbox{Im}\left[\mathcal{H}(\xi)\right] &=&\displaystyle  \int_{\xi}^1\frac{dy}{y}
  \left[\left(\frac{1}{1-y}\right)_++\delta(1-y)\left(\frac{3}{2}-\ln(2)\right)
  \right]H^{(+)}\left(\frac{\xi}{y}\right)\\
\\
&=&\displaystyle  \int_{\xi}^1
  \frac{dy}{1-y}\left[\frac1yH^{(+)}\left(\frac{\xi}{y}\right)-H^{(+)}\left(\xi\right)\right]+\left(\frac{3}{2}-\ln(2)+\ln(1-\xi)\right) H^{(+)}\left(\xi\right)\,.
\end{array}
\end{equation}















\end{document}
