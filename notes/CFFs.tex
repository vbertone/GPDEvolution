%% LyX 2.0.3 created this file.  For more info, see http://www.lyx.org/.
%% Do not edit unless you really know what you are doing.
\documentclass[twoside,english]{paper}
\usepackage{lmodern}
\renewcommand{\ttdefault}{lmodern}
\usepackage[T1]{fontenc}
\usepackage[latin9]{inputenc}
\usepackage[a4paper]{geometry}
\geometry{verbose,tmargin=3cm,bmargin=2.5cm,lmargin=2cm,rmargin=2cm}
\usepackage{color}
\usepackage{babel}
\usepackage{float}
\usepackage{bm}
\usepackage{amsthm}
\usepackage{amsmath}
\usepackage{amssymb}
\usepackage{graphicx}
\usepackage{esint}
\usepackage[unicode=true,pdfusetitle,
 bookmarks=true,bookmarksnumbered=false,bookmarksopen=false,
 breaklinks=false,pdfborder={0 0 0},backref=false,colorlinks=false]
 {hyperref}
\usepackage{breakurl}
\usepackage{mathrsfs}

\makeatletter

%%%%%%%%%%%%%%%%%%%%%%%%%%%%%% LyX specific LaTeX commands.
%% Because html converters don't know tabularnewline
\providecommand{\tabularnewline}{\\}

%%%%%%%%%%%%%%%%%%%%%%%%%%%%%% Textclass specific LaTeX commands.
\numberwithin{equation}{section}
\numberwithin{figure}{section}

%%%%%%%%%%%%%%%%%%%%%%%%%%%%%% User specified LaTeX commands.
\usepackage{babel}

\@ifundefined{showcaptionsetup}{}{%
 \PassOptionsToPackage{caption=false}{subfig}}
\usepackage{subfig}
\makeatother

\begin{document}

\section{Compton forms factor with evolution}

The factorised expression for Compton form factors (CFF) takes the
following form:
\begin{equation}
    \mathcal{H}(\xi, t, Q^2) = \int_{-1}^1
    \frac{\mathrm{d}x}{\xi}\sum_{a= q,g} T^a\left(\frac{x}{\xi},
      \frac{Q^2}{\mu^2},
      \alpha_s(\mu^2)\right)H^a(x,\xi,t,\mu^2)\equiv \sum_{a= q,g}
    T^a(Q,\mu)\otimes H^a(\mu)\,.
    \label{eq:convol}
\end{equation}
For the sake of the argument, we assume that there is one single quark
generation and that $H^{q}$ corresponds to the singlet
combination(\footnote{In the presence of more quark generations, also
  a non-singlet component has to be considered that, to $O(\alpha_s)$,
  multiplies the same hard cross section $T^q$ as the singlet but
  evolves multiplicatively through the evolution kernel
  $K_{qq}$.}). This allows us write the CFF is a matricial form:
\begin{equation}
\mathcal{H}(Q) = (T^{q}(Q,\mu)\quad T^{g}(Q,\mu))\otimes {H^{q}(\mu) \choose
  H^{g}(\mu)}\,.
\label{eq:CFFvect}
\end{equation}
The hard cross sections $T^q$ and $T^g$ admit a perturbative expansion
whose truncation at $\mathcal{O}(\alpha_s)$ reads:
\begin{equation}
  T^{q}(Q,\mu) = T_0^q + \alpha_s(\mu)\left(T_1^q+T_{\rm
      coll}^q\log\frac{\mu^2}{Q^2}\right)\,\quad\mbox{and}\quad T^{g}(Q,\mu) = \alpha_s(\mu)\left(T_1^g+T_{\rm
      coll}^g\log\frac{\mu^2}{Q^2}\right) \,.
\end{equation}
We know that the evolution of GPDs $H^{q}$ and $H^{g}$ is governed by
the following RG equation:
\begin{equation}
\frac{d}{d\ln\mu^2} {H^{q}(\mu) \choose H^{g}(\mu)} = \begin{pmatrix}
K_{qq}(\mu) & K_{qg}(\mu)\\
K_{gq}(\mu) & K_{gg}(\mu)
\end{pmatrix}\otimes {H^{q}(\mu) \choose H^{g}(\mu)}\,,
\label{eq:eveq}
\end{equation}
where the evolution kernels $K_{ab}$ obey the perturbative expansion:
\begin{equation}
K_{ab}(\mu) = \alpha_s(\mu)\sum_{n=0}\alpha_s(\mu) K_{ab}^{(n)}\,.
\end{equation}
Assuming to know GPDs at some initial scale $\mu_0$, the solution to
Eq.~(\ref{eq:eveq}) can be written as:
\begin{equation}
{H^{q}(\mu) \choose H^{g}(\mu)} = \begin{pmatrix}
\Gamma_{qq}(\mu,\mu_0) & \Gamma_{qg}(\mu,\mu_0)\\
\Gamma_{gq}(\mu,\mu_0) & \Gamma_{gg}(\mu,\mu_0)
\end{pmatrix}\otimes {H^{q}(\mu_0) \choose H^{g}(\mu_0)}\,,
\label{eq:solDGLAP}
\end{equation}
where we have defined the evolution operator as:
\begin{equation}\label{eq:evolop}
\begin{pmatrix}
\Gamma_{qq}(\mu,\mu_0) & \Gamma_{qg}(\mu,\mu_0)\\
\Gamma_{gq}(\mu,\mu_0) & \Gamma_{gg}(\mu,\mu_0)
\end{pmatrix}
= \exp\left[\int_{\mu_0}^{\mu} d\ln\mu'^2\begin{pmatrix}
K_{qq}(\mu') & K_{qg}(\mu')\\
K_{gq}(\mu') & K_{gg}(\mu')
\end{pmatrix}\right]
\end{equation}
where the exponential function has to be interpreted as a path-ordered
exponential. Given the exponential form of the evolution operator, it
should clear that the following equality holds:
\begin{equation}
\begin{pmatrix}
\Gamma_{qq}(\mu,\mu_0) & \Gamma_{qg}(\mu,\mu_0)\\
\Gamma_{gq}(\mu,\mu_0) & \Gamma_{gg}(\mu,\mu_0)
\end{pmatrix}
= 
\begin{pmatrix}
\Gamma_{qq}(\mu,Q) & \Gamma_{qg}(\mu,Q)\\
\Gamma_{gq}(\mu,Q) & \Gamma_{gg}(\mu,Q)
\end{pmatrix}\otimes
\begin{pmatrix}
\Gamma_{qq}(Q,\mu_0) & \Gamma_{qg}(Q,\mu_0)\\
\Gamma_{gq}(Q,\mu_0) & \Gamma_{gg}(Q,\mu_0)
\end{pmatrix}\,.
\label{eq:breaking}
\end{equation}
Now, if the scales $\mu$ and $Q$ are not too far apart, the first
evolution operator in the r.h.s. of the equation above can be
systematically expended in powers of $\alpha_s$. It is easy to see
that to $\mathcal{O}(\alpha_s)$ the expansion is:
\begin{equation}
\begin{pmatrix}
\Gamma_{qq}(\mu,Q) & \Gamma_{qg}(\mu,Q)\\
\Gamma_{gq}(\mu,Q) & \Gamma_{gg}(\mu,Q)
\end{pmatrix}
= \begin{pmatrix}
1 & 0\\
0 & 1
\end{pmatrix}+\alpha_s(\mu)\begin{pmatrix}
K_{qq}^{(0)} & K_{qg}^{(0)}\\
K_{gq}^{(0)} & K_{gg}^{(0)}
\end{pmatrix}\ln\frac{\mu^2}{Q^2}+\mathcal{O}(\alpha_s^2)\,.
\label{eq:expop}
\end{equation}
We now replace the GPD vector at the final scale $\mu$ in
Eq.~(\ref{eq:CFFvect}) with that at the initial scale $\mu_0$ using
Eqs.~(\ref{eq:solDGLAP}), (\ref{eq:breaking}), and~(\ref{eq:expop}):
\begin{equation}
\mathcal{H}(Q) = (T^{q}(Q,\mu)\quad T^{g}(Q,\mu)) \otimes 
\left[\begin{pmatrix}
1 & 0\\
0 & 1
\end{pmatrix}+\alpha_s(\mu)\begin{pmatrix}
K_{qq}^{(0)} & K_{qg}^{(0)}\\
K_{gq}^{(0)} & K_{gg}^{(0)}
\end{pmatrix}\ln\frac{\mu^2}{Q^2}\right]\otimes
\begin{pmatrix}
\Gamma_{qq}(Q,\mu_0) & \Gamma_{qg}(Q,\mu_0)\\
\Gamma_{gq}(Q,\mu_0) & \Gamma_{gg}(Q,\mu_0)
\end{pmatrix}
\otimes {H^{q}(\mu_0) \choose H^{g}(\mu_0)}\,.
\end{equation}
The first two terms in the r.h.s. of the equation above can be
combined and only terms up to $\mathcal{O}(\alpha_s)$ retained:
\begin{equation}
\begin{array}{c}
\displaystyle \left\{(T^{q}(Q,\mu)\quad T^{g}(Q,\mu)) \otimes 
\left[\begin{pmatrix}
1 & 0\\
0 & 1
\end{pmatrix}+\alpha_s(\mu)\begin{pmatrix}
K_{qq}^{(0)} & K_{qg}^{(0)}\\
K_{gq}^{(0)} & K_{gg}^{(0)}
\end{pmatrix}\ln\frac{\mu^2}{Q^2}\right]\right\}^T\\
\\
=\displaystyle {T_0^{q}\choose T_0^{g}} + \alpha_s\left[{T_1^{q}\choose
  T_1^{g}}+{T_{\rm coll}^{q}+T_0^{q}\otimes
  K_{qq}^{(0)}\choose T_{\rm coll}^{g}+T_0^{q}\otimes K_{qg}^{(0)} }\ln\frac{\mu^2}{Q^2}\right]+\mathcal{O}(\alpha_s^2)\,.
\end{array}
\end{equation}
In order for the CFF $\mathcal{H}$ to be independent from the
normalisation scale $\mu$ up to $\mathcal{O}(\alpha_s)$, we need to
require:
\begin{equation}
\begin{array}{rcl}
T_{\rm coll}^{q}&=&-T_0^{q}\otimes K_{qq}^{(0)}\\
\\
T_{\rm coll}^{g}&=&-T_0^{q}\otimes K_{qg}^{(0)}
\end{array}\,.
\end{equation}
Finally one has:
\begin{equation}
  T^{q}(Q,\mu) = T_0^q + \alpha_s(\mu)\left(T_1^q -T_0^{q}\otimes K_{qq}^{(0)}\log\frac{\mu^2}{Q^2}\right)\,\quad\mbox{and}\quad T^{g}(Q,\mu) = \alpha_s(\mu)\left(T_1^g -T_0^{q}\otimes K_{qg}^{(0)}\log\frac{\mu^2}{Q^2}\right)\,,
\end{equation}
so that:
\begin{equation}
\mathcal{H}(Q) = \left[(T_0^{q}\quad 0) +\alpha_s(Q)(T_1^{q}\quad T_0^{g})\right]\otimes 
\begin{pmatrix}
\Gamma_{qq}(Q,\mu_0) & \Gamma_{qg}(Q,\mu_0)\\
\Gamma_{gq}(Q,\mu_0) & \Gamma_{gg}(Q,\mu_0)
\end{pmatrix}
\otimes {H^{q}(\mu_0) \choose H^{g}(\mu_0)}\,.
\end{equation}
Now, let us also assume that $\mu_0$ and $Q$ are not too far
apart. This allows us to expand the evolution operator between $\mu_0$
and $Q$ as above:
\begin{equation}
\begin{pmatrix}
\Gamma_{qq}(\mu,Q) & \Gamma_{qg}(\mu,Q)\\
\Gamma_{gq}(\mu,Q) & \Gamma_{gg}(\mu,Q)
\end{pmatrix}
= \begin{pmatrix}
1 & 0\\
0 & 1
\end{pmatrix}+\alpha_s(Q)\begin{pmatrix}
K_{qq}^{(0)} & K_{qg}^{(0)}\\
K_{gq}^{(0)} & K_{gg}^{(0)}
\end{pmatrix}\ln\frac{Q^2}{\mu_0^2}+\mathcal{O}(\alpha_s^2)\,.
\end{equation}
Let us also assume that $H^g(\mu_0)=0$. This finally leads to:
\begin{equation}
\begin{array}{rcl}
\mathcal{H}(Q) &=& \displaystyle T_0^{q}\otimes
H^{q}(\mu_0)+\alpha_s(Q)\left[ T_1^{q}\otimes H^{q}(\mu_0) +
                   T_0^{q}\otimes K_{qq}^{(0)}\otimes H^{q}(\mu_0)
                   \ln\frac{Q^2}{\mu_0^2}\right]+\mathcal{O}(\alpha_s^2)\\
\\
&=&\displaystyle T_0^{q}\otimes
H^{q}(\mu_0)+\alpha_s(Q)\left[ T_1^{q}\otimes H^{q}(\mu_0) -
                   T_{\rm coll}^{q}\otimes H^{q}(\mu_0)
                   \ln\frac{Q^2}{\mu_0^2}\right]+\mathcal{O}(\alpha_s^2)\,.
\end{array}
\end{equation}
In conclusion, up to $\mathcal{O}(\alpha_s)$, assuming that the gluon
GPD is identically zero at the initial scale $\mu_0$, and for scales
$Q$ not too far from $\mu_0$, a shadow GDP needs to simultaneously
fulfil the following equalities:
\begin{equation}
\begin{array}{rcl}
T_0^{q}\otimes H^{q}(\mu_0) &=& 0\,\\
\\
T_1^{q}\otimes H^{q}(\mu_0) &=& 0\,\\
\\
T_{\rm coll}^{q}\otimes H^{q}(\mu_0) &=& - T_0^{q}\otimes
                                         K_{qq}^{(0)}\otimes
                                         H^{q}(\mu_0) = 0\,.
\end{array}
\end{equation}
Conjecture: in order to fulfil the last equalities, it would be enough
to require:
\begin{equation}
  K_{qq}^{(0)}\otimes H^{q}(\mu_0) = 0\,.
\end{equation}
If this equality holds, I believe that $H^{q}(\mu_0)$ is unaffected by
the evolution, not only at $\mathcal{O}(\alpha_s)$, but to all
orders. This would allow us to relax the restriction $Q\simeq
\mu_0$. This can be checked numerically.

\subsection{Scale variations}

The computation of a CFF in terms of coefficient functions, evolution
operator, and initial-scale GPDs can be schematically written as:
\begin{equation}\label{eq:convolutionsketch}
\mathcal{H}(Q,\mu,\mu_0) = \mathbf{T}^T(Q,\mu)\otimes {\bm
  \Gamma}(\mu,\mu_0)\otimes \mathbf{H}(\mu_0)
\end{equation}
where all convolution products also imply a matrix product, with the
$\mathbf{T}$ and $\mathbf{H}$ being the column vectors of coefficient
functions and GPDs, and ${\bm \Gamma}$ the evolution matrix. All the
irrelevant arguments are dropped. $\mathbf{H}(\mu_0)$ are the input
GPDs at some \textit{fixed} initial scale $\mu_0$. The evolution
operator ${\bm \Gamma}(\mu,\mu_0)$, given in Eq.~(\ref{eq:evolop}),
has the role of resumming terms of the kind
$\alpha_s^m(\mu)\ln^n(\mu/\mu_0)$ to all orders, with $m = n$ being
the leading-logarithm (LL) accuracy, $m=n-1$ next-to-leading logarithm
(NLL), and so on. Finally, the coefficient functions
$\mathbf{T}(Q,\mu)$ can be regarded as a perturbative expansion in
powers of $\alpha_s(\mu)$ whose truncation determines the fixed-order
perturbative accuracy. For the specific case of a CFF, truncation at
$\mathcal{O}(1)$ is said leading-order (LO) approximation, truncation
at $\mathcal{O}(\alpha_s)$ next-to-leading order (NLO), and so on.
Generalising the discussion above, the coefficient of the contribution
to $\mathbf{T}(Q,\mu)$ proportional to $\alpha_s^n(\mu)$,
$\mathbf{T}_n$, has the following general structure:
\begin{equation}
\mathbf{T}_n(\mu,Q) = \sum_{k=0}^{n}\mathbf{T}_n^{(k)}\ln^k\left(\frac{\mu}{Q}\right)\,.
\end{equation}
While the term $\mathbf{T}_n^{(0)}$ requires the calculation on the
$n$-loop diagrams, all other coefficients $\mathbf{T}_n^{(k)}$, with
$k\geq 1$, are combinations of $\mathbf{T}_m^{(0)}$, with $m<n$, and
the perturbative coefficient of the evolution kernels
$\mathbf{K}^{(l)}$, with $l<n$.(\footnote{In principle, also the
  coefficients of the $\beta$-function are present, but since in the
  following we are interested to NLO accuracy where these coefficients
  are not present yet, we will not need to discuss them.}) This is a
direct consequence of the fact that the coefficients
$\mathbf{T}_n^{(k)}$, with $k\geq 1$, are specifically engineered to
compensate the effect of the evolution beween $\mu$ and $Q$ up to
order $n$ inclusive. The compensation is such that, if the coefficient
functions $\mathbf{T}_n$ are computed to N$^{n}$LO and the evolution
to N$^{m}$LL with $m\geq n-1$, variations of $\mu$ in
Eq.~(\ref{eq:convolutionsketch}) will generate terms whose size is at
worst of order $\mathcal{O}(\alpha_s^{n+1}(\mu)\ln^{n+1}(\mu/Q))$. If
we require $\mu\simeq Q$, such that $\ln(\mu/Q)\simeq 1$, the
logarithms will no longer inflate the terms thus generated leaving us
with $\mathcal{O}(\alpha_s^{n+1})$. An operational way of writing this
statement is the following: given two different factorisation scales
$\mu$ and $\nu$, both in the vicinity of $Q$, the following relation
holds:
\begin{equation}\label{eq:convolutionsketchdiff}
\mathcal{H}(Q,\mu,\mu_0) - \mathcal{H}(Q,\nu,\mu_0) = C\times \alpha_s^{n+1}(\mu)\,.
\end{equation}
where $\mathcal{H}$ is computed at N$^{n}$LO with N$^{n-1}$LL (or more
accurate) evolution, and $C$ is a constant of order one approximately
proportional to $\ln^{n+1}(\mu/\nu)$. Notice that the argument in the
r.h.s. of the equation above is set to $\mu$. In fact, this is
arbitrary. One could have chosen any scale of order $Q$ in that this
would give rise to subleading differences in $\alpha_s$. A possible
choice that eliminates one of the scales is of course $\nu = Q$.

Now the question is: how does one probe numerically that
Eq.~(\ref{eq:convolutionsketchdiff}) is effectively true? My
suggestion is the following: let us fix the scales $\mu_0$, $Q$ and
$\mu$. For example $\mu_0=1$~GeV, $Q=30$~GeV,
$\mu=60$~GeV.(\footnote{Despite $\mu_0$, $Q$ and $\mu_0$ are linearly
  equally spaced, their ratios, $\mu/\mu_0=60$ and $\mu/Q=2$, are such
  that the interval $[\mu_0,\mu]$ requires resummation while $[Q,\mu]$
  does not.}). Since $\mu\simeq M_Z$, \textit{i.e.} $\ln(\mu/M_Z)\simeq 1$,
one can write:
\begin{equation}\label{eq:implem}
\mathcal{H}(Q,\mu,\mu_0) - \mathcal{H}(Q,Q,\mu_0) = C'\times \alpha_s^{n+1}(M_Z)\,.
\end{equation}
The reason for choosing $M_Z$ as a reference scale is that often the
evolution of $\alpha_s$ is computed using the value of $\alpha_s(M_Z)$
as a boundary condition (the specific value is typically around
$\alpha_s(M_Z)= 0.118$). A way of testing Eq.~(\ref{eq:implem}) is to
numerically compute the r.h.s. changing the value of $\alpha_s(M_Z)$
used as a reference for the evolution of the coupling in a reasonable
range (\textit{e.g.} $\alpha_s(M_Z)\in [0.05, 0.2]$) and to check that
it scales like $\alpha_s^{n+1}(M_Z)$. The same exercise can be
repeated picking different values of $Q$ and $\mu$. By doing so, one
should observe that the value of the coefficient $C'$ extracted with
different pairs $(Q,\mu)$ scales like $\ln^{n+1}(\mu/Q)$.

Of course, it is not necessary to use $\alpha_s(M_Z)$ as a scaling
parameter. One can use $\alpha_s(\mu)$ as prescribed by
Eq.~(\ref{eq:convolutionsketchdiff}) and this value changed by moving
the position of the Landau pole.



\end{document}
